\chapter{Conclusion}
\label{chap:conclusion}

%Conclusion goes here  (2-3 pages)
%\begin{itemize}
%\item Status of implementation – know limitations and weaknesses (functionality as well as
%technical solution)
%\item Reflection on project work (aka project retrospective)
%\item  Features that should be added in future versions of the application
%\item  Technical solutions that should be modified, replaced, or added in future version of
%the application; technologies that should be considered for future versions of the
%application
%\end{itemize}


\section{Status of implementation, limitations and weaknesses}
The quiz is done as we originaly inteded it. We planned for more features outside the quiz, but the core functionlity was completed as intended. I was really great to accually complete what we planned, espesially afther falling really short on our plans last semester. There is one exeltion this this ohverever. We never added the special area, but it was not clear either if that will ever be used. It works on many ways exactly the same as competiton groups.

The current implementation is not perfect ohvever as mentions as it lacckes a bit in the testing department, both manuel user testing and automatic scripted testing. We have not really had any problems from not having automatic testing bit we think we have now reached the point where eny further works requiers it.

\section{Reflection on project work}
Is was fun, but we sort of miss help from mariuz, and starge that he both abandended us and took another group on a payed trip to poland...


\section{Features to be added}
Even if the core quiz is ready we still have a lot of functionality we want to add in. The first in probably the speial area where organisations can puchase a special are/page on the site with is own subset of species for a custum quiz. this migth hovever be unnessasarily as in really overlaps with competition groups. Only the future will tell what it will become in the end.

Another feature really requested is the videoquiz where people can for example take a quiz showing videos of birds and then determening the specie of them. This will require a lot more storage that the sever currently have as videos are a lot larger then images and sounds. Some people preffer observing birds more life like setting, as in the video. It will still just be one of many choices on how to take the quiz and the user own choice is till the most important.

The system does currentlu has a hint system build in where the user can either remove choices or get the first lethers of the speciename. This are just the start as we want o add even more ways of helping the user and giving them extra types of hints. One type of hints coulkd be gining additional facts about the correct  specie like its size, weigth and qwingspan. Maby even some markers. It could also be possible to click a button to get another image or sound of the specie to help identifdy it. The possiblities arer almost endless 
Expanding on the last part about hints we would also want to futher improve on the gamification elements in the system without harmin the intrinsic motication of the user to learn. This could be a really challeging task as addign gamification can easily destroy intrinsic motivation due to the overjustification effect and may only work for a short time due to the novelty factor (https://www.researchgate.net/publication/256743509_Does_Gamification_Work_-_A_Literature_Review_of_Empirical_Studies_on_Gamification). There is great potential for improvement, but this is an merging field and we should tread carefully not to hurt what we already have. 

One potential problem with the system currently is that is requiers manually inputting all the choices for any question somewhat amnually. There is some automatication, butr the system could benefit by having amode where the choices/markers are uatomaticly generated. This could be done by values already in the system like simular birds by aperence and sound. This could then  be used as a secoud difficulity to the media difficulity. Harder difficulities would have more simualr choices. This could also be extendet to hints by having a button that makes the coices easier. One problem by that hint system hovever is that theu user can just select the on that does not change.

improving UX



addign automaticly generated alternatives






\section{Technical solutionsthat should be improved}

improving build prosess and concating fidles
