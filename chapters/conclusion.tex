\chapter{Conclusion}
\label{chap:conclusion}

%Conclusion goes here  (2-3 pages)
%\begin{itemize}
%\item Status of implementation – know limitations and weaknesses (functionality as well as
%technical solution)
%\item Reflection on project work (aka project retrospective)
%\item  Features that should be added in future versions of the application
%\item  Technical solutions that should be modified, replaced, or added in future version of
%the application; technologies that should be considered for future versions of the
%application
%\end{itemize}


\section{Status of implementation, limitations and weaknesses}
The quiz is completed as we originally intended. We planned for more features outside the quiz, but the core functionality was completed as intended. It was really great to actually complete what we planned, especially after falling really short on our plans last semester.However there is one exception to this, we never added the special area, but it was not clear  if it will ever be used. It works on many ways exactly the same as competition groups.

The current implementation is not perfect, as mentioned it lacks a bit in the testing department, both manual user testing and automatic scripted testing. We have not really had any problems from not having automatic testing bit we think we have now reached the point where any further work would require it.

\section{Reflection on project work}
(aka project retrospective)

We selected a wide range of new technologies for this project which contributed to a rather steep learning cure in the beginning as nobody really was familiar with anything. The first few weeks were therefore promptly spent learning everything so we could get to a point where we could start working. From there work speed steadily improved as our velocity chart should have shown if we had consistent length on sprints. We did however decide to focus more on always having something to do rather than sprints with fixed length and requiring all tasks to be done before the next. Development speed reached really high speed towards the end with 1/3 all commits being in the last sprint. This is due to most of them being really small and the team getting really familiar with the technologies. 

There was a real challenge in getting all team members working efficiently since we all did come from different backgrounds and learned in different ways. Despite some setbacks we did manage to get everyone to the same point by the end of the project. This type of experience with working on a initially divided team is really valuable as any team out in the real world will be like that at some point. Experiencing this diversity is very relevant experience that later on we can apply to a real job. The work was  structured as most program development projects with a focus on always learning something new and focusing on what benefits the user while using the latest in project organisation applications, like Jira.

%Angular 2 turned out to be the best

Working together at school was probably the best thing we did. The group dynamics was great where we in one moment was discussing a task and in the next debating the cultural differences. The social factor is just as important as being able to instantly discuss tasks as problems occur. Development time was clearly shorter when we worked together because it was much easier to explain tasks and challenges while visually showing what we where discussing. Remote only teams have some huge challenges here, especially getting each team member on the same page. For keeping a consistent vision of what the application/product is. We experienced how is it to work with some members of the team being away for a while. It is a valuable lesson learned on how the task should be clearly defined in Jira and how to communicate efficiently. 

In the beginning of the project the communication with the main supervisor was going really well, we had meetings and got valuable feedback and suggestions on the project but as the project progressed it became almost impossible to get a meting with the main supervisor.
At the end (last 3 sprints) we repeatedly asked for help with the project and we were turned down each time. This was the time when we need help the most. 
Is was fun, but we sort of miss help from main supervisor, at stage that he both abandoned us and took another group on a payed trip to Poland.

\section{Features to be added}
Even if the core quiz is ready we still have a lot of functionality we want to add in. The first in probably the special area where organisations can purchase a special area/page on the site with is own subset of species for a custom quiz. This might however be unnecessary as it really overlaps with competition groups. Only the future will tell what it will become in the end.

Another feature really requested is the video quiz where people can for example take a quiz showing videos of birds and then determining the specie of them. This will require a lot more storage that the sever currently have as videos are a lot larger then images and sounds. Some people prefer observing birds in more life like setting, as in the video. It will still just be one of many choices on how to take the quiz and the user own choice is till the most important.

The system currently has a hint system build in where the user can either remove choices or get the first letters of the specie name. This is just the start, we want to add even more ways of helping the user and giving them extra types of hints. One type of hints could be getting additional facts about the correct specie like its size, weight and wingspan. It could also be possible to click a button to get another image or sound of the specie to help identify it. The possibilities are almost endless.

Expanding on the last part about hints we would also want to further improve on the gamification elements in the system without harming the intrinsic motivation of the user to learn. This could be a really challenging task as adding gamification can easily destroy intrinsic motivation due to the over justification effect and may only work for a short time due to the novelty factor\cite{DoesG74:online}. There is great potential for improvement, but this is an merging field and we should tread carefully not to hurt what we already have. 

One potential problem with the system currently is that is requires manually inputting all the choices for any question somewhat manually. There is some automation, but the system could benefit by having mode where the choices/markers are automatically generated. This could be done by values already in the system like similar birds by appearance and sound. This could then  be used as a second difficulty to the media difficulty. Harder difficulties would have more similar choices. This could also be extended to hints by having a button that makes the choices easier. One problem with this kind of hints system however is that the user can just select the ones that does not change when asking for easier choices on the particular question.

The design and user experience of the current application need some improvements currently. A lot of functionality are not really self explanatory as well as not having the best design either. This is mostly due to no team member currently being a designer and what we made is more in the way of programmers design. Working, but far from perfect. One example is that the user can get overloaded with information in both additional settings and the actual quiz. For example, in the quiz there is an entire table showing all the quiz settings. This table should be either removed or completely redesigned. A solution is hiding it by default, but allowing users to see it by clicking a button like "show quiz settings". Another issue is how the list of available languages are in English, making it hard to select your own language if you do not know it in English, somewhat defeating the purpose of having a language select. Another problem regarding the area list as it is always in English no mater the active language. Most of these problems, if not all  should be possible to fix with further development testing and designing. 


\section{Technical solutions that should be improved}

The current build proses during development is entirely automatic using Gulp and lite server with node package manager (npm) for package management. However it is not perfect when it comes to deployment. It does not concatenate any files and requires the upload of all individual compiled files including all node dependencies. It is entirely possible to both concatenate all node dependencies and project files into one or two files of needed (for each file type like CSS and JS). We did not have time to implement this, but would really like to do this if we had the time. It would enormously speed up deployment and speed up application speed with less to transfer and load.

The system currently pulls all translation when it starts up. This can be significantly improved by doing local caching using local storage or indexed DB. It could fetch the translations the first time the app runs and then just ask for changes from the server. This could decrease load on the sever and amount of data transmitted, but not really improve app load time. This is due to a lot of other simultaneously server request are preformed on start up anyway. Some of these can be cached like the list of areas, but still competition groups needs to be loaded as well as the specie list and language list. All of these could also be cached, but just checking for changes is usually just as consuming time wise as just getting all the data. Also storing that much data on mobile devices will eat away from already really precious storage.

The project is in need of some cleanup and refactoring. Some services are fragmented and should be combined into one like change language service and translation puller. They are overlapping a lot and would greatly benefit from merging into one. This project has been developed in an iterative approach which led to a lot of unused code as new features were added. This should be cleaned up and maybe refactored in order to keep the project tidy and neat for further development. This project will most likely have a long lifespan with with the customer and having the code at its best for as long as possible will greatly help us in the future.
