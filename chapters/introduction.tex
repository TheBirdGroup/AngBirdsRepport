
\chapter{Introduction}
\label{chap:introduction}
 
%\begin{itemize}
%\item Visualizing the statistics with charts
%\end{itemize}
% this is example of how bullet points are made
 
\par The main purpose of this project is to translate the existing application from Flash Technology to Angular 2 and to add new features into this application.
The goal of our quiz platform is to teach people to recognize birds by their appearance and their singing. Application is also adapted to mobile devices and is made in user friendly manner. Prototype of this application passed usability tests, and final version of the application took into account users preferences and comments.
\par
Project work was concentrated mostly on front-end. Additionally some changes were made in the back-end. The group is consisted of 4 members. The responsible for back-end  was Jan Greger, all members were working on the front-end. Component (Modular??) structure of Angular 2 framework helped us to divide the work in efficient way - working on the same time on different components. 
\par
The current version of the application uses Flash Technology. To highlight the problem with using Flash we can provide some of the disadvantages of this technology. \newline
Some of the disadvantages are shown below:
\par
1. Long time to load pages. We know how it is irritating to wait while page is loaded. 
\par
2. Watching video- websites that use Flash technology requires users to have Flash installed
\par
3. Difficulties with optimization for search engines. 
\par
4. Accessibility. For using on mobile devices Flash is probably not the best choice. A lot of mobile phones and tablets cannot access Flash websites.  \cite{Advan78:online}
\par
Taking into account all problems that were described above, we can conclude that using state-of-the-art technology will increase efficiency, quality and usability of application. 
\par
The application has already stakeholders, there is already institution that is interested(using??) in our product. Nord University has official course that is concentrated on learning birds and they currently use old version of application. 
\par
The uniqueness and value of this project is in his universality, the web application is flexible, which means that it is easy to modify it for different purposes. That means that our product can be used not only for learning birds but for learning mammals, butterflies, plants or whatever. This fact extend number of potential customers a lot. 
\par
This report is divided into 5 parts. In next chapters you will find information about application functionality and technologies and system architecture. We will describe key features of the application, assess functional strengths and weaknesses, innovativeness. We will provide system architecture and description of main components. We will mention what future improvement can we do in further work part. The report will be summarised by conclusion. 





   